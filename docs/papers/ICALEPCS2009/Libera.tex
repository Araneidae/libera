\documentclass{JAC2003}

\usepackage{graphicx}
\usepackage{booktabs}

\usepackage{mathptmx}
\usepackage{amsmath}

\usepackage{url}
\usepackage{mdwlist}

% Using Xy-pic for our diagrams
\usepackage[dvips,matrix,arrow,curve,line,all]{xy}
\renewcommand{\objectstyle}{\displaystyle}
\renewcommand{\labelstyle}{\textstyle}


% Seems we're allowed to steal a bit of title space if we need to.
\setlength{\titleblockheight}{27mm}

%% Tone down hyphenation enthusiasm
\sloppy
\hyphenpenalty 2000



\begin{document}
\title{%
    THE DIAMOND LIGHT SOURCE CONTROL SYSTEM INTERFACE \\
    TO THE LIBERA ELECTRON BEAM POSITION MONITORS}

\author{%
    M.G.~Abbott, G.~Rehm, I.S.~Uzun,
    Diamond Light Source, Oxfordshire, UK}

\maketitle

\begin{abstract}
    Libera Electron Beam Position Monitors (EBPMs) used at Diamond provide
    information about the electron beam position at a variety of frequency
    scales from 10\,MHz, through revolution frequency, down to 10\,Hz.
    Diamond Light Source has implemented an EPICS interface to Libera
    integrating all of this information into the overall control system. In
    conjunction with the timing and fast orbit feedback interfaces this
    provides access to the rich data sets and functionality provided by
    Libera. The details of the interfaces and available data, both directly
    from the Libera and through a concentrator, are described here.
\end{abstract}




\section{Libera System Overview}

Diamond Light Source uses 168 Electron Beam Position Monitors (EBPMs) in the
storage ring, together with 22 in the booster and 7 in each of the two transfer
paths, to monitor the position of the electron beam.

Each EBPM consists of a button block in the vacuum vessel with four pickup
electrodes surrounding the electron beam; these pick up electric fields
corresponding to the electron beam position as modulations of the machine RF
($f_{RF} \approx 499.65\,$MHz).  The four button signals are fed by coaxial
cables to a Libera\cite{i-tech} EBPM processor where the amplitudes of the
signals are processed and used to compute the electron beam position, as shown
in figure \ref{buttons}.

\begin{figure}[h]
\centering
$
% Command to draw and label a single button
\newgraphescape{b}#1#2#3{
    "z"[#2#3(1.5)]   % Move into buttton position
    [l(0.8)]-[r(1.6)] [l(0.8)]-[#2(1.5)] % Draw the button body
    [#2]*{#1} % Label the button
}
\vcenter{
\xygraph{
    !{0;<0.7em,0em>:0}
    % Remember the origin in the centre
    []="z"
    % Draw the four buttons
    !bAul !bBur !bCdr !bDdl
    % Draw the vessel outline
    "z"[u(2)l(3)] ="a" "z"[d(2)r(3)] ="c"
        !{ "a"."c"*\frm<1em>{-} }
    % Draw the centre beam and axes
    "z"*+{\circ} ="o" :[u(4)]*+{y} "o":[l(5)]*+{x}
}
}
\qquad
\begin{aligned}
    x &= K_X \cdot \frac{A-B-C+D}{A+B+C+D} + X_0 \\[0.4em]
    y &= K_Y \cdot \frac{A+B-C-D}{A+B+C+D} + Y_0 \\[0.4em]
    I &= K_I \cdot (A+B+C+D) / G
\end{aligned}
$

\caption{%
    Computing electron beam position $x,y$ and current $I$ from measured
    button intensities $A$, $B$, $C$, $D$, scaling factors $K_X$, $K_Y$
    determined by button geometry, offsets $X_0$, $Y_0$ measured by beam
    based alignment, current scaling $K_I$ dynamically calibrated against
    a DCCT, and the currently programmed RF gain $G$.
}
\label{buttons}
\end{figure}

Libera processes these signals in three stages: RF board, FPGA, and embedded
controller.  In the RF board the four signals pass through a cross-bar switch
into four parallel analogue channels where the signal is amplified with
programmable gain and band-pass filtered to approximately 10\,MHz around
$f_{RF}$ before being undersampled by ADCs at around 117\,MHz (sample
frequency $f_S$).  The sampling frequency is carefully chosen to both place
the intermediate frequency (IF) close to $\frac14 f_S$ and avoid, as far as
possible, the folding back of IF and machine revolution ($f_{R} = f_{RF}/936$)
harmonics onto the intermediate frequency.

The raw sampled signal is processed by a Virtex-II Pro FPGA with firmware
developed by the manufacturer, Instrumentation Technologies (I-Tech), with
some modifications by Diamond.  In the FPGA the raw signal is first
demultiplexed according to the cross-bar switch setting and phase and
amplitude corrected to compensate for channel differences before being mixed
with a digital local oscillator to bring the intermediate frequency down to
DC: this involves the conversion of button signals to quadrature $I,Q$ values.
These signals are filtered by repeated stages of digital down conversion to
reduce the sample frequencies to the values provided by Libera, as below.

\begin{description*}
\renewcommand{\makelabel}{}     % Abusing the description style a bit here.
\item
    $f_S \approx 117$\,MHz (bandwidth 10\,MHz).  Fixed length 1024 point
    waveforms of raw ADC button signals captured on trigger are useful for
    first turn and transfer path position measurement.
\item
    $f_R \approx 534$\,kHz.  A circular buffer of two seconds of turn-by-turn
    data is stored, from which triggered waveforms of at least one second of
    turn-by-turn position data can reliably be provided.  Waveforms are read
    from this buffer on every trigger, and the same buffer is read on
    postmortem trigger.
\item
    $f_{FA} \approx 10$\,kHz.  Electron beam positions are generated at
    approximately 10\,kHz and used for beam position validation (machine
    protection interlock) and are transmitted to the Fast Feedback
    network\cite{fast-feedback}.  This is the Fast Acquisition (FA) data
    stream.
\item
    $f_{SA} \approx 10$\,Hz.  Beam positions at approximately 10\,Hz can be
    handled directly by EPICS clients as scalars and are made available as
    10\,Hz updates over Ethernet.  This is the Slow Acquisition (SA) data
    stream.
\end{description*}

The turn-by-turn data stream can also be decimated by the FPGA by a factor of
64 on readout; this is useful for cycling machines such as the Diamond
booster, where a complete ramp can be returned as a 3,000 point waveform.

The FA data stream is available via the fast feedback network; the remaining
signals are processed by the final part of the system, the embedded
controller.


\begin{figure*}[tb]
\centering
% Figure to illustrate overall structure of internal Libera processing chain.
{
% Draws the ADC symbol at the current position
\newgraphescape{a}#1{
    []!{\save   \xygraph{ 
        []="o"              % Remember origin on entry
        !{+0;+<1em,0em>:} % Set appropriate scaling
        !~-{@[thicker]@{-}}     % Thicker lines
        "o"[l(2)u]="a"[d]-[ur]-[r(2.5)]-[dd]="b"-[l(2.5)]-[ul]
        !~-{@{-}}               % and back to normal
        "o"*{\text{#1}} 
    } \restore} 
    % Put origin at point of entry and combine with opposite corners
    !{"o"."a"."b"} 
}
% Draws a box with a formula inside it at the current position
\newgraphescape{t}#1{
    []!{\save   \xygraph{ 
        []="o"
        !{+0;+<1em,0em>:}
        !~-{@[thicker]@{-}}     % Thicker lines
        "o"[l(1.75)u]="a"-[r(3.5)]-[d(2)]="b"-[l(3.5)]-[u(2)]
        !~-{@{-}}               % and back to normal
        "o"*{#1} 
    } \restore} 
    % Put origin at point of entry and combine with opposite corners
    !{"o"."a"."b"} 
}
% Labels a row with the arguments set in a two column table
\newgraphescape{l}#1{
    [r(0.2)]
    []*+!!<0pt,\fontdimen22\textfont2>!L{\text{
        \begin{tabular}[c]{@{\ttfamily}p{6mm}l}
        #1
        \end{tabular}
    }}
}
% Draws simple text with the proper baseline
\newgraphescape{s}#1{
    []*+!!<0pt,\fontdimen22\textfont2>{\text{#1}}
}
% Draw a single button
\newgraphescape{b}#1#2{
    "z"[#2(1.5)#1]="o"   % Move into buttton position
    [l(0.8)]-[r(1.6)] "o"-[#1(1.5)][] % Draw the button body
}
% The figure itself
\hfill\xygraph{
    []!{0;<3pc,0pc>:<0pc,2.2pc>::}
    % 
    % Label the columns of the diagra
    [ll] !s{\textbf{Buttons}}
    [r(2.5)] !s{\textbf{FPGA Processing}}
    [r(5.5)]  !s{\textbf{EPICS Process Variable Interface by function}}
    \\[u(.2)]
    %
    % CK label
    [rrr]="ck" !l{CK & Clock control} \\[u(.2)]
    %
    % Draws the button block assembly 
    [ll]="z" !{\save\xygraph{ 
        !{+0;+<0.5em,0em>:} 
%        !~-{@[thicker]@{-}}     % Thicker lines
        !bul !bur !bdr !bdl 
        "z"[u(2)l(3)]="a" "z"[d(2)r(3)]="c" !{"a"."c"*\frm<1em>{-}}
    } \restore}
    !{"z"."a"."c"}="buttons"
    % followed by the ADC
    && !a{ADC}="adc"  
    %
    % Sample frequency data feed
    "buttons"-"adc"-[r]="nadc":[rr]^{f_S \approx 117\,\text{MHz}} 
    !l{FT & 1024\,pt waveform + scalars on trigger (5\,Hz)}
    \\[d(.2)]
    %
    % Turn by turn data feed
    !t{\div 220}="d220"-[r]="n220":[rr]^{f_R \approx 533\,\text{kHz}} 
    !l{
        FR & 2048\,pt waveform on trigger (5\,Hz) \\
        TT & Up to 524,288\,pt waveform on demand \\
        PM & 16,384\,pt waveform on postmortem trigger \\
        SC & Signal Conditioning
    }
    !{*--\frm{\{}}
    \\[d(.2)]
    %
    % BN decimated data feed
    !t{\div 64}="d64":@{-->}[rrr] [ll]:@{}[rr]^{\text{(on demand)}}
    !l{BN & Decimated waveform on trigger (5\,Hz)}
    \\
    %
    % FA data feed and interlocks
    !t{\div 53}="d53"-[r]="n53":@{.>}[rr]^{f_{FA} \approx 10\,\text{kHz}}
    !l{
        FF & Fast feedback status and control \\
        IL & Interlock control \\
        MS & Mean current measured between triggers
    }
    !{*--\frm{\{}}
    \\
    %
    % SA data feed
    !t{\div 1024}="d1024"-[r]:[rr]^{f_{SA} \approx 10\,\text{Hz}}
    !l{SA & Scalars at 10\,Hz} \\
    %
    % The last annotations
    [r(0.5)]!s{Configuration and control}
    [r(2.5)] !l{
        CF & Configuration \\
        SE & Sensors and system monitoring \\
        VE & Version identification
    }
    !{*--\frm{\{}}
    %
    % Finally the curves joining things together.
    "ck"-[lll]:"adc"
    % Original lines without bendy corners
%    "nadc"-[d(.6)]-[ll]-[d(.6)]-"d220"     % The original working version
%    "n220"-[d(.6)]-[ll]-[d(.6)]="b64"-"d64" "b64"-[d]-"d53"
%    "n53"-[d(.5)]-[ll]-[d(.5)]-"d1024"
    %
    % More complicated bendy corners
    %    This is rather horrible really
    "nadc"!{\ar@{-} 
        `d/4pt -(0,.6) `^d-(2,0)!/r4pt/ `^r-(0,.6)!/u4pt/ "d220" }
    "n220"!{\ar@{-}
        `d/4pt -(0,.6) `^d-(2,0)!/r4pt/ -(0,.6)="b64" 
        `^r-(0,1.6)!/u4pt/ "d53"}
    "b64"-"d64"
    "n53"!{\ar@{-}
        `d/4pt +(0,-.5) `^d+(-2,0)!/r4pt/ `^r+(0,-0.5)!/u4pt/ "d1024" }
%
}
}

\caption{%
    Overview of Libera system with data processing chains and associated
    groups of PVs. The two letter function code is part of the PV name,
    for example \texttt{SR01C-DI-EBPM-01:SA:X} names an SA PV on the first
    storage ring Libera.}
\label{system-overview}
\end{figure*}


\section{System Software}

For its control system interface Libera uses an ARM XScale PXA 255 processor
running at 400\,MHz with 32\,MB flash and 64\,MB RAM.  There is no hardware
floating point support: this must be allowed for during software development.
The software on the controller consists of embedded Linux and a kernel device
driver, both provided by I-Tech, together with Diamond developed processes
adding EPICS functionality and supplementing or replacing I-Tech's original
software\cite{libera-epics-web}.

The direct software interface to the FPGA on Libera consists of device nodes
provided by the kernel driver, and some directly accessed registers.  The
control interface to Libera is provided by the EPICS driver which provides a
complete interface to all of the functionality available from Libera and hides
the kernel driver and registers.  Data access and control is provided via
Process Variables\footnote{%
%
    An EPICS Process Variable (PV) is typically a single numerical value, or a
    fixed length waveform, with a fixed name which can be read or written
    by any remotely connected EPICS Channel Access client.  The naming
    convention for PVs is an important part of controller design.
%
}, a fundamental notion in any EPICS controller.

The EPICS driver provides around 500 PVs grouped into 14 major functions as
shown in figure \ref{system-overview}.  The system software is structured as
three daemons: clock control, temperature control, and an EPICS IO
controller (IOC).


\subsection{Clock Control}

The ADC sample clock is maintained at a precise ratio of the machine
revolution clock frequency via a voltage controlled crystal oscillator (VCXO)
and a phase measurement circuit in the FPGA which counts sample clocks every
53,382 revolution clocks, roughly every 100ms.

Stability of the ADC clock is essential for good beam position measurement:
the clock daemon is able to maintain the phase of the sample clock to within
$\pm 1.5$ samples against the machine clock; this is sufficient for stable
position measurement and synchronisation.

The sample clock is maintained at a frequency close to an integer fraction of
the revolution clock (220 samples per revolution, versus 936 electron bunches
per revolution).  However, if $f_S$ is precisely 220 samples per revolution
then beating with harmonics of revolution sidebands is seen in the filtered
beam position, so it is necessary to ``detune'' the sample clock; at Diamond
the frequency offset is approximately 4\,kHz.

The clock controller is also important for the fast feedback communication
network which requires all Liberas to be synchronously clocked (to within a
microsecond or so).  Synchronisation is done via a global trigger, and the
clock controller checks for loss of synchronisation.


\subsection{Position Measurements}

The EPICS driver processes the button intensity data and computes positions
from both raw ADC rate data and filtered turn-by-turn waveforms.  For speed of
computation the calculation of button magnitudes from $I,Q$ values is
performed using the CORDIC\cite{cordic} algorithm, and a specialised fast
division routine using Newton-Raphson was written to accelerate position
calculation.

For ``first turn'' (\texttt{FT}) or transfer path position measurement the raw
ADC waveform is frequency shifted and filtered in software, and a user
defined window defines the region of interest from which a scalar position is
calculated.

Two triggers are used for turn-by-turn waveform capture: a global system
trigger, normally fired on injection into the storage ring every 200ms, and a
postmortem trigger fired on loss of machine protection interlock.  Postmortem
data is very valuable in diagnosing causes of beam loss.

Three groups of PVs are used for turn-by-turn data: \texttt{FR} and
\texttt{BN} PVs update on every trigger returning fixed length waveforms,
while the \texttt{TT} PVs provide variable length triggered waveform capture
designed for machine physics applications.  The postmortem \texttt{PM} group
also returns turn-by-turn data.


\subsection{Signal Conditioning}

The beam position calculation is very sensitive to small variations in the
measured signal: disturbing the coaxial button cables can move the measured
position by microns.  In particular, drifting and instabilities in the four
analogue channels in the RF board cause high levels of lower frequency noise.
This is compensated in Libera via an RF cross-bar which ensures that each
button signal is passed, in turn, through each filter and amplifier chain on
the RF board; the final measured button signal is then an average of the four
channels.

This switching process largely eliminates low frequency drift in measured
position, but at the cost of high levels of noise at the switching frequency.
At Diamond switching occurs every 40 turns with a complete cycle of switching
every 320 turns (Diamond's RF boards require 8 switch positions), resulting in
a lowest switching frequency of 1.67\,kHz.  Careful filter design plus a pair
of notch filters removes most of this switching noise from FA and SA data, but
it is important to reduce the impact of switching.

One source of noise is the different phase and gain of each analogue channel.
This can be modelled as a complex scaling factor $K_c$ for each channel $c$,
so the measured signal with the cross-bar switch in position $n$ passing
button $b$ through channel $p(n,b)$ is $Z_{n,b} = K_{p(n,b)} X_b$.  By
measuring $\vec{Z}$ for all button positions and channels (a 2048 point
turn-by-turn waveform is ample for this measurement) the input signal
$\vec{X}$ can be estimated by averaging over all switch positions, assuming
that the input signal is constant during measurement, and from this the error
$\vec{K}$ can be computed\footnote{%
%
    This is oversimplified, see the code\cite{libera-epics-web} for details.
    In practice the estimation of $\vec{X}$ requires geometric means of
    magnitudes and arithmetic means of angles, and $K_c$ also needs to depend
    on $n$.
%
}.  By working with complex gains and signals, phase and gain can be
compensated together at the sampled intermediate frequency.

The Diamond EPICS driver performs this calculation every few seconds during
normal operation and updates a two point FIR filter for each channel to
perform the correction, and also provides all the measurements as PVs.
Monitoring the signal conditioning PVs can provide very helpful information
about the health of the RF board.


\subsection{Interlocks and Monitoring}

Within the FPGA the 10\,kHz data stream is used as a beam position interlock
to protect against mis-steering of the stored beam: when the stored beam is
more than 10\,mA the beam is confined to a window of $\pm 1$\,mm; if this
window is exceeded a machine protection interlock is dropped and the beam is
dumped.

PVs are provided for monitoring the health of the Libera, and an alarm is
indicated if appropriate.  Monitors include fan speeds and internal system
temperatures, on board voltages, available RAM (including log file
consumption) and CPU usage.  The health of the clocks is monitored, and loss
of synchronisation generates an alarm.


\section{System Integration}

All 168 storage ring Liberas are operated together through control scripts,
from client applications, or through the ``Concentrator'': this EPICS server
monitors all Liberas, gathers position readings into aggregate waveforms, and
manages global functions.  The Concentrator further monitors the maximum ADC
levels on all Liberas and controls the gain programmed into the RF board to
avoid ADC overflow in any Libera.  This is done globally to avoid jumps in
position from unexpected changes in gain, particularly during fast feedback.

When the beam intensity measurement from all Liberas is averaged the result is
a very sensitive beam current measurement with high frequency noise levels
well below those of the DCCT used for beam current measurement.  By
calibrating this average against the more accurate DCCT it is possible to
measure very low storage ring injection currents with high precision and
sensitivity.

The 10\,kHz Fast Acquisition data stream from each Libera passes onto the fast
feedback network which communicates real time position data from all attached
Liberas to all attached nodes, including fast feedback motor
controllers\cite{fast-feedback}.  An extra ``sniffer'' node is attached to
this network and manages a circular buffer of 100,000 samples of beam
positions around the ring at 10\,kHz which can be read out in blocks via
EPICS.  Using this tool detailed spectral analysis of full machine beam
behaviour can be investigated from 1.5\,kHz down to 100\,mHz.


\section{Conclusions}

EPICS integration of Libera into the Diamond control system has been
successful and Libera operates reliably.

The Diamond Libera EPICS driver has been developed as an open source
application under GPL and is available for download from Diamond's web
site\cite{libera-epics-web}.  This driver is in use at synchrotron light
sources in Germany, China and Korea, and is being evaluated for use at NSLS2
in the USA.

Future work on Libera will mostly concentrate on minor changes, but there are
two major developments in the pipeline: a new rootfs built from scratch, based
on Busybox and a standard Linux kernel, which we'll be deploying to the
machine this year; and plans for moving much of the kernel driver into user
space to improve maintainability.




\begin{thebibliography}{9}

\bibitem{i-tech}
Instrumentation Technologies, \url{http://www.i-tech.si}.

\bibitem{fast-feedback}
M.G.~Abbott, J.A.~Dobbing, M.T.~Heron, G.~Rehm, J.~Rowland, I.S.~Uzun,
S.~Duncan, ``Diamond Light Source Electron Beam Position Feedback: Design,
Realization and Performance'', ICALEPCS'09.


\bibitem{libera-epics-web}
Diamond Libera EPICS driver downloads:
\url{http://controls.diamond.ac.uk/downloads/libera/}.

\bibitem{cordic}
Volder, J.E., 1959; ``The CORDIC Trigonometric Computing Technique'', IRE
Transactions on Electronic Computers, V.~EC-8, No.~3, pp.~330-334.

\end{thebibliography}

\end{document}
