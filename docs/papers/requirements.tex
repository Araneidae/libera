\documentclass[
    a4paper,
    fleqn
]{article}

\usepackage{mathptmx}        % Postscript fonts
\usepackage{amsmath}
\usepackage{amssymb}
\usepackage{array}
\usepackage{tabularx}

\sloppy
\hyphenpenalty 4000


\newcommand{\id}[1]{\texttt{#1}}
\newcommand{\XY}{\id{X}, \id{Y}}
\newcommand{\SQ}{\id{S}, \id{Q}}
\newcommand{\XYSQ}{\XY, \SQ}
\newcommand{\ABCD}{\id{A}, \id{B}, \id{C}, \id{D}}
\newcommand{\ABCDIQ}{%
  \id{AI}, \id{AQ}, \id{BI}, \id{BQ}, 
  \id{CI}, \id{CQ}, \id{DI}, \id{DQ}}


\newenvironment{epics}{%
  \begin{list}{}{%
    \newcommand{\entry}[2]{\item[##1\hfill]\hfill##2\\}%
  }%
}{%
  \end{list}%
}


\begin{document}
\title{Requirements for Libera EPICS Interface}
\author{Michael Abbott}
\date{29\textsuperscript{th} July 2005}
\maketitle

\section{Overview}

This note documents the requirements for EPICS support for the Libera
electron beam position monitoring system.  

Note that this document is out of date and should only be treated as a
historical document.

\subsection{Libera Signal Processing Chain}

The raw data captured by Libera before digital processing is in the
form of individual button signals for buttons labelled A, B, C and D.
This data stream is sampled at approximately 117MHz and carries the
underlying machine 500MHz clock at an aliased frequency of
approximately 30MHz.

In other words, immediately after digitisation all the beam position
information is carried in the sidebands of a frequency line at
approximately 30MHz: this is the \emph{intermediate frequency}.

This raw signal is then immediately mixed with a oscillator running at
the same intermediate frequency.  This shifts the beam position
information down to close to DC.  The output from this step is a pair
of signals in quadrature for each button.

Unfortunately there are significant difficulties in phase locking the
mixing clock (at low signal levels the phase locking would be very
vulnerable to noise, so it would be necessary to feed back a filtered
signal) and it is also not practical to reduce the quadrature pair of
signals back to an absolute value, as this is a costly operation.
Thus at high frequencies we have eight signals to deal with.

Subsequent processing reduces the sample rate from the raw sample rate
to turn-by-turn sampling at either 1.893MHz (for the booster) or
534kHz (for the storage ring), and further processing reduces this to
4kHz samples (for fast feedback), 10Hz samples (for slow feedback) and
other custom sample rates.

In the present implementation of Libera only turn-by-turn quadrature
data is available together with the same reduced in sample rate by 64.
Future versions of Libera should provide turn-by-turn data reduced to
X, Y, S, Q together with the other data streams described in this
document.

\subsection{Forms of Libera Data}

The underlying data is button signals, \ABCD.  This is often provided
in quadrature form, \ABCDIQ, in which case the button signals have to
be calculated as
\begin{align*}
\id{A} &= \sqrt{\id{AI}^2+\id{AQ}^2} &
\id{B} &= \sqrt{\id{BI}^2+\id{BQ}^2} \\
\id{C} &= \sqrt{\id{CI}^2+\id{CQ}^2} &
\id{D} &= \sqrt{\id{DI}^2+\id{DQ}^2} 
\enspace.
\end{align*}

The end user form of the Libera data is as positions \XY{} which is
supplemented by a signal strength value \id{S} and a ``quality
factor'' \id{Q}.  These are calculated thus:
\begin{align*}
\id{S} &= \id{A} + \id{B} + \id{C} + \id{D}  &
\id{X} &= 
    K_X \times (\id{A} - \id{B} - \id{C} + \id{D}) / \id{S} - X_0 \\
\id{Q} &= (\id{A} - \id{B} + \id{C} - \id{D}) / \id{S} &
\id{Y} &= 
    K_Y \times (\id{A} + \id{B} - \id{C} - \id{D}) / \id{S} - Y_0
\enspace.
\end{align*}
The strength value \id{S} together with the programmed gain can be
used provide a measurement of beam current.  The quality factor \id{Q}
should be close to zero and can be alarmed to detect faults in the BPM
system.

It is planned to process and deliver all data as 32-bit integers with
positions \XY{} scaled to nanometres.


\section{Requirements}

The Libera beam position monitors operate in three quite different
environments.
\begin{description}
\item[Transfer paths.]  Electrons travel through the transfer paths in
  transient bursts lasting no more than half a microsecond; this
  repeats at 5Hz.
\item[Booster.]  Electrons stay in the booster for 100ms during which
  time their positions may change quite considerably;  a detailed view
  of this is needed.  This also repeats at 5Hz.
\item[Storage.]  Electrons stay in the storage ring indefinitely.
\end{description}




\subsection{Transfer Paths (LtB, BtS)}

Electrons pass through the transfer path in a single pulse lasting for
between one bunch ($\ll$2ns) up to one booster turn (528ns).  In both
cases this corresponds to one sample at the turn-by-turn clock, which
in both cases will be the booster clock.  At the raw sampling rate of
approximately 117MHz these times correspond to between 1 to 62
samples, though the impulse response of the system will stretch this
somewhat.

The following data can usefully be derived from this signal and
published through the EPICS interface.
\begin{itemize}
\item
  Reduction of the triggered turn-by-turn data to a single point to be
  provided as \XYSQ{} and \ABCD.  This can be done by capturing a
  small window around the trigger point and averaging.
\item
  It may be useful to provide access to the turn-by-turn waveform used
  in the averaging process above.  This is a very short waveform,
  perhaps 5 to 10 points.
\item
  If possible it would be useful to have access to the raw waveform as
  captured by the ADC.  If so a waveform length of 100 seems suitable.
  This will not be available from Libera for some time.
\end{itemize}
All of these data points would update at 5Hz in response to the
appropriate injection trigger.  The volume of data involved is very
small and no special tune testing operation is required.


\subsection{Booster (BR)}

A booster cycle lasts 100ms and is repeated at 5Hz.  During this time
turn-by-turn will accumulate nearly 200,000 samples --- access to this
data is required for tune measurement.  During normal operation it
will also be desirable to provide access to a reduced form of this
waveform.  During commissioning it will be helpful to have access to
the ``first turn'' data.

Thus we will provide the following through EPICS.
\begin{itemize}
\item
  Position waveforms providing \XYSQ{} for a complete booster cycle
  with a reduced sample rate of, say, 2kHz --- this means each
  waveform would be 200 points long.
\item
  Averaged positions from the start, middle and end of the booster
  cycle: again, \XYSQ{} would be provided as single points for each
  time. 
\item
  Tune measurement.  In this mode a single full turn-by-turn waveform
  would be captured in its entirety from a single booster cycle and
  then read out in segments.

  Data reduction to \XY{} and \SQ{} may be done in the Libera or in a
  supporting Matlab routine.  The choice of which implementation to
  use will be done to optimise performance.
\item
  First turn data.  This can be accessed either through the tune
  measurement interface or through the same interface as used for
  transfer paths.
\end{itemize}

During normal operation all data will update at 5Hz.  Tune testing
will involve single-shot triggering and slower readout.


\subsection{Storage (SR)}

The storage ring operates continuously.  Three main activities need to
be supported:
\begin{itemize}
\item
  Continuous position monitoring at 10Hz.
\item
  Fast feedback status monitoring and control.
\item
  Tune measurement.
\end{itemize}
The interface to the fast feedback system is not covered here, however
we do provide access to the fast feedback position buffer.

The following functionality is provided through EPICS:
\begin{itemize}
\item
  Positions reduced to \XYSQ{} updated at 10Hz.  This will be provided
  at all times.
\item
  Tune measurement.  In response to a trigger a long waveform of at
  least 25,000 points will be captured at turn-by-turn and made
  available for subsequent readout.  It should be possible to use a
  subArray record to provide access to the entire waveform.  

  This requirement can be unified with booster tune measurement.
\item
  Fast feedback buffer.  Positions reduced to \XY{} at 4kHz sample
  rate provided as a continually updating waveform.
\end{itemize}


\subsection{Configuration and Control}

There are a handful of configuration settings and controls.  The
following list is likely to grow.

\begin{itemize}
\item
  Fast feedback interfacing: to be defined.
\item
  Clock synchronisation.  A process involving close interaction with the
  timing system is required for Libera clock synchronisation.  It
  would make sense to provide an EPICS interface for this.
\item
  Attenuators and RF cross-bar switch.  These can either be
  automatically controlled by Libera or managed through an EPICS
  interface.
\item
  Scaling coefficients and zero point.  The scaling coefficients
  $K_X$, $K_Y$ and zero point $X_0$, $Y_0$ will be defined when the
  beam is aligned with the dipole centres\footnote{It's easy to see
  how $X_0$, $Y_0$ are determined: but how are $K_X$ and $K_Y$
  found?}.
\item
  Internal temperature and fan speeds can be monitored through the
  EPICS interface.
\item
  Support for vxStats style monitoring of processor activity will be
  useful.  It would be good to provide the same records so the same UI
  can be used for vxWorks IOCs and Libera.
\end{itemize}


\section{EPICS Interface}

The EPICS interface can be divided into six functional groups.  Which
functional group is active in which beam position monitor position is
shown by the table below.
\begin{figure}[h]
\begin{tabular}{>{\tt}llccc}
 & & Transfer & Booster & Storage \\
FT & First Turn & \checkmark &  \checkmark &  \checkmark \\
TUNE & Tune Measurement & & \checkmark &  \checkmark \\
RAMP & Booster Ramp & & \checkmark & \\
10HZ & Storage 10Hz & & & \checkmark \\
FAST & Fast Feedback & & & \checkmark \\
CT & Control \& Miscellaneous  & \checkmark &  \checkmark &  \checkmark 
\end{tabular}
\end{figure}

In the EPICS interface descriptions below the record names are
described concisely.  Each functional group has a 2 or four character
abbreviation listed in the table above and in the title of the section
below.  This is combined with the short form of the record name as
listed and the location of the beam position monitor on the machine to
produce the final record name.

For example, the first turn X position is listed with name \id{X}
under First Turn \texttt{:FT:}.  The corresponding record name for the
first beam position monitor in the linac to booster transfer path is
\begin{quote}
\texttt{LB-DI-EBPM-01:FT:X}
\end{quote}



\subsection{First Turn \texttt{:FT:}}

These records support the analysis of the behaviour of the electron
beam immediately after a trigger.

\begin{epics}
\entry{\XYSQ, \ABCD}{longin}
  First turn positions and associated averaged button measurements.
  Calculated from triggered turn-by-turn data.
\entry{\id{WFA}, \id{WFB}, \id{WFC}, \id{WFD}}{waveform(10)}
  Triggered turn-by-turn button values.
\entry{\id{RAWA}, \id{RAWB}, \id{RAWC}, \id{RAWD}}{waveform(100)}
  Raw triggered data at underlying 117MHz sample rate.  Not available
  until this data is provided by Libera.
\end{epics}


\subsection{Tune Measurement \texttt{:TUNE:}}

These records provide access to long triggered waveforms at
turn-by-turn sample rates. 

\begin{epics}
\entry{\id{LEN}}{longout}
  Specifies number of points to be captured in one trigger.  At least
  200,000 can be specified.
\entry{\id{ARM}}{bo}
  Writing 1 to this record arms the trigger and enables the capture of
  one waveform of the specified length.
\entry{\id{READY}}{bi}
  This bit is reset when \id{ARM} is set and is set when data is
  available to be read out from the captured buffer.
\entry{\id{WFX}, \id{WFY}, \id{WFS}, \id{WFQ}}{subArray(10000)}
  These records provide a window into the captured data.  Up to 10,000
  points can be read directly, or data can be read by moving the
  subArray through the captured waveform.
\end{epics}


\subsection{Booster Ramp \texttt{:RAMP:}}

Access to low bandwidth view of the booster ramp data: both full
waveforms and points at the start, middle and end of the ramp are provided.

\begin{epics}
\entry{\id{WFX}, \id{WFY}, \id{WFS}, \id{WFQ}}{waveform(200)}
  Booster positions at sample rate of 2kHz covering entire booster
  cycle from initial trigger.
\entry{\id{STAX}, \id{STAY}, \id{STAS}, \id{STAQ}}{longin}
  Average booster position at start of cycle.
\entry{\id{MIDX}, \id{MIDY}, \id{MIDS}, \id{MIDQ}}{longin}
  Average booster position at middle of cycle.
\entry{\id{ENDX}, \id{ENDY}, \id{ENDS}, \id{ENDQ}}{longin}
  Average booster position at end of cycle.
\end{epics}


\subsection{Storage 10Hz \texttt{:10HZ:}}

Storage ring 10Hz position updates only.

\begin{epics}
\entry{\XYSQ}{longin}
  Beam positions reduced to 10Hz bandwidth, updating at 10Hz.
\end{epics}


\subsection{Fast Feedback \texttt{:FAST:}}

Access to 4kHz positions being fed into the fast feedback system.

\begin{epics}
\entry{\id{WFX}, \id{WFY}, \id{WFS}, \id{WFQ}}{waveform(4000)}
  Fast feedback waveforms, updating at 1Hz.  It should be possible to
  capture consecutive updates to each of these waveforms to generate a
  continuous stream of 4kHz position data.
\end{epics}


\subsection{Control \& Miscellaneous \texttt{:CT:}}

Controls for attenuation and switches, zero points and scaling, and
general status readouts.

\begin{epics}
\entry{\id{ATT1}, \id{ATT2}}{longout}
  Settings for first and second stages of attenuation.  Attenuation is
  set uniformly for all channels, but the two stages of attenuation on
  each channel are separately controllable.
\entry{\id{AGC}}{bo}
  Enable or disable automatic gain control.  If \id{AGC} is clear the
  attenuator settings can be manually set by controlling \id{ATT1} and
  \id{ATT2}; when this bit is set the attenuation is automatically
  controlled in the FPGA.
\entry{\id{GAIN}}{longin}
  Returns the current gain.
\entry{\id{XBAR}}{longout}
  Sets the crossbar switch.  One of 16 positions can be selected.
  This only takes effect if the crossbar is in manual mode.
\entry{\id{XBARAUT}}{bo}
  Controls automatic crossbar switch operation.
\entry{\id{K\_X}, \id{K\_Y}, \id{X\_0}, \id{Y\_0}}{longout}
  Calibration factors and zero offset points for \XY{} calculations.
\end{epics}




\end{document}
